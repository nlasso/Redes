En el trabajo práctico trabajamos analizando las rutas que siguen los paquetes para llegar a distintos destinos de nuestro interés. Para lograr dicho objetivo, desarrollamos una herramienta que realiza un echo request al destino pero incrementando el valor TTL del paquete ICMP de a una unidad. De esta forma los distintos hosts que el paquete transita nos devuelven un \textit{Time Exceeded} como respuesta junto con su IP, haciendo posible, además de su identificación y un orden de los mismos, un calculo de RTT. Midiendo diferencias significativas en el valor de RTT logramos identificar los enlaces submarinos que conectan al mundo entre sí.

En el intento de medir hasta que host llega el paquete ICMP con determinado TTL o si llega a destino, nos encontramos con que algunos de los hosts no respondían nada. Esto resulta de la configuración de los mismos que por medidas de seguridad no permiten, o bien responder paquetes o bien filtran paquetes ICMP. Estos hosts no fueron incluidos en el analisis de las rutas y los RTT finales de cada Hop incluyen el tiempo que toma transitar por Hosts que no registran estos paquetes. Como conclusión a esto, hoy en día no se puede fiar del uso de los paquetes ICMP para realizar traceroutes dado que no todos los routers están configurados para tratar con ellos. Por esta razón, creemos que utilizar paquetes TCP sería de mayor precisión para obtener todos los hops de una ruta aunque habría que constatarlo experimentalmente.

Nos encontramos con resultados interesantes cuando analizamos los datos. Por ejemplo, cuando intentamos realizar la gelocalización de los distintos IPs. En ese caso, existen IPs que tienen un rango correspondiente a un país específico pero físicamente estan ubicados en otras regiones del mundo, lo que dificultó la ubicación de los mismos y la creación de los distintos mapas. Por otro lado, mediante la detección de cambios de ruta, pudimos observar que países como India tienen una alta redundancia en sus rutas (o un problema en su red), ya que se detectaron muchos cambios de rutas al realizar las repeticiones. Ademas, notamos que habian muchas universidades que no responden a un echo-request (por ejemplo la Universidad de Tokyo, Japón) por lo que se dificultó su selección.

Por otro lado, pudimos comprobar empíricamente la relación proporcional entre el RTT y la distancia geográfica de las rutas, y su relación inversa con el valor estimado por el throghput obtenido por la ecuación de Mathis, salvando algunas inconsistencias menores.

Descubrimos diferentes incosistencias entre las tablas de geolocalización IP, y tuvimos que tomar información de varias hasta encontrar datos consistentes con nuestros resultaods.

En un trabajo futuro, sería buena idea analizar las distintas rutas que puede tomar un paquete para llegar a destino, exponiendo los lugares por los que transita y los tiempos de cada uno de los caminos. También queda pendiente realizar este mismo análisis y otros utilizando paquetes TCP para mayor precisión.