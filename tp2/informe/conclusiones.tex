Pudimos comprobar empíricamente la relación directamente proporcional entre el RTT y la distancia geográfica de las rutas, y su relación inversa con el valor estimado por el throughput obtenido por la ecuación de Mathis.

~

El umbral propuesto pareciera identificar correctamente los enlaces submarinos que conectan al mundo entre sí, pero también arroja siempre como ``falso positivo'' el enlace Buenos Aires - Miami. De todos modos, dada su gran distancia y la geografía de América, es probable que dicho enlace sea también submarino (aunque seguramente no sea trasatlántico) por lo que no descartamos de lleno el método propuesto para identificarlos.

~

En el intento de medir hasta qué host llega el paquete ICMP con determinado TTL o si llega a destino, nos encontramos con que algunos de los hosts no respondían nada. Esto resulta de la configuración de los mismos que por medidas de seguridad no permiten, o bien responder paquetes o bien filtran paquetes ICMP. Estos hosts no fueron incluidos en el analisis de las rutas y los RTT finales de cada hop incluyen el tiempo que toma transitar por Hosts que no registran estos paquetes. Como conclusión a esto, hoy en día el uso de los paquetes ICMP para realizar traceroutes no parece fiable, dado que no todos los routers están configurados para tratar con ellos. Por esta razón creemos que utilizar paquetes TCP sería de mayor precisión para obtener todos los hops de una ruta, aunque habría que constatarlo experimentalmente.

~

Nos encontramos con resultados interesantes cuando analizamos los datos, por ejemplo cuando intentamos realizar la gelocalización de los distintos IPs. Existen IPs que tienen un rango correspondiente a un país específico pero físicamente estan ubicados en otras regiones del mundo, lo que dificultó la ubicación de los mismos y la creación de los distintos mapas. Por otro lado, mediante la detección de cambios de ruta, pudimos observar que países como India tienen una alta redundancia en sus rutas (o un problema en su red), ya que se detectaron muchos cambios de rutas al realizar las repeticiones. Ademas, notamos que habia muchas universidades que no responden a un echo-request (por ejemplo la Universidad de Tokyo, Japón) por lo que se dificultó su selección.

~

En un trabajo futuro, sería buena idea analizar las distintas rutas que puede tomar un paquete para llegar a destino, exponiendo los lugares por los que transita y los tiempos de cada uno de los caminos. También queda pendiente realizar este mismo análisis y otros utilizando paquetes TCP con algún puerto que sepamos que se encuentra abierto hasta el host de destino para mayor precisión.