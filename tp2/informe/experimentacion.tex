Durante la experimentación encontramos ciertas anomalías detectadas al correr los experimentos. Las mismas fueron:

\begin{itemize}
    \item Hops que no respondían. En base a este problema se generó la hipótesis de que había hops en los cuales los paquetes ICMP enviados llegaban con un \textit{Time-Exceeded} a un router que estaba configurado para no responder a estos paquetes o para filtrarlos directamente. En base a esto se configuró el método \textit{SR1()} de Scapy para que dado cierto \textit{Timeout} el mismo dejara de esperar a que el router le respondiera permitiéndonos aumentar el TTL y llegar al siguiente Hop.
    
    \item Corridas en las cuales no se llegaba a ningún destino haciendo que el script nunca terminara incluso aumentando el TTL y el tiempo de Timeout. Con la misma hipótesis que la anterior, realizando experimentaciones a diferentes universidades, hubo algunas, como las que se muestran a continuación, cuyo servidor estaba configurado para no responder. Por ejemplo:
    \begin{itemize}
        \item Universidad de Tokio
        \item Melbourne University
        \item University of Mumbay
    \end{itemize}
    En este caso, lo que se realizó para constatar este comportamiento previamente fue hacer un \textit{PING()} a los destinos para corroborar si los mismos responden. De esta manera se supo si había muchos hops que no respondían, como sucedía anteriormente, entre la IP de origen y la de destino o si era efectivamente el destino.

    \item Las rutas al mismo destino cambiaban. Al correr muchas veces a la misma IP de destino notamos que las rutas podían variar, con lo cual no era lo suficientemente fiable el cálculo del RTT. Dado esto, implementamos una verificación para descartar una corrida cuando se detectaba que una ruta no era igual a la primer ruta obtenida. De esta forma se aseguró la misma ruta en todos los casos. Otra solución posible hubiese sido enviar los paquetes directamente a los hops una vez fijada la ruta, pero preferimos no utilizar esta alternativa ya que la ruta directa hacia cada uno de los hops puede ser muy disinta a la que nos llevaba a ellos cuando nuestro destino era otro (corroborado en pruebas preliminares).
    
    \item RTTs no lineales, es decir, si bien se supone que a medida que se realizan saltos a hops más lejanos el RTT debería aumentar, esto no siempre sucedía. Al realizar las experimentaciones se notó que para cierto rango de IPs que pertenecen a la misma región la diferencia entre los RTTs podía ser negativa. Esto puede deberse a que muchos routers encolan las respuestas de ICMP de modo tal que tardan mucho más en responderlas que en reenviarlas. Para confirmarlo o descartarlo, como trabajo futuro se podría implementar el mismo \textit{traceroute} pero con paquetes TCP.
    
    \item Otra anomalía ocurría cuando para hops muy distantes la diferencia era muy pequeña. Tras revisar las locaciones de las IPs llegamos a la conclusión de que, si bien hay rangos IPs que pertenecen a la misma región y que deberían estar relativamente cercanos, sucede que la IP física se encuentra realmente en otra locación provocando que se produzca más de un salto transatlántico y que los valores de RTT varíen de esta manera. Estos casos se pueden apreciar en la sección de resultados.    

\end{itemize}

Las universidades a las que finalmente se les realizó el traceroute fueron:

\begin{itemize}
	\item Universidad de Berkeley, California, EEUU. Con la dirección: berkeley.edu
	\item Perm State University, Rusia. Con la dirección: en.psu.ru
	\item Iceland University, Iceland. Con la dirección: english.hi.is
	\item Cochin University of Scienc and Technology, Cochin, India. Con la dirección: cusat.ac.in
	\item University of Pretoria, Pretoria, South Africa. Con la dirección: up.ac.za 
\end{itemize}

Dichas universidades respondieron a los PING previos para realizar la experimentación y todas pertenecen a distintos continentes.

El throughput fue estimado mediante la ecuación de Mathis, descripta en la sección de implementación. Para elegir el valor del $\alpha$ y de la cantidad de iteraciones de la ecuación que calcula el EstimatedRTT ($n$) realizamos pruebas con $\alpha =$ 0,1; 0,2; ... ; 1 y $n =$ 10, 20, ... , 50.