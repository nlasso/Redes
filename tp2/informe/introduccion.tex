En el presente trabajo práctico tuvimos como desafío implementar una herramienta muy común de diagnóstico de red, el traceroute.
Por medio de la misma los administradores de red pueden analizar la ruta que sigue un paquete hasta un destino dado.

Luego se nos pidió usar la herramienta diseñada para evaluar las rutas que toman los paquetes hacia diferentes lugares del mundo, incluyendo un análisis sobre cómo distinguir los enlaces trasatlánticos y evaluando su correcitud posteriormente con herramientas de geolocalización de IP.

Por último se nos pidió estimar el valor del $throughput$ de la conexión para las rutas analizadas, teniendo en cuenta la pérdida de paquetes.

