En el presente trabajo práctico tuvimos como desafío implementar una herramienta muy común de diagnóstico de red, el traceroute.

Por medio de la misma los administradores de red pueden analizar la ruta que sigue un paquete hasta un destino dado.

Los fundamentos de dicha herramienta son muy sencillos.
Se utilizan paquetes IP para recorrer la ruta aumentando de un salto a la vez.
Esto se hace por medio del campo TTL (\textit{Time to live}) que nos permite configurar la cantidad máxima de ``saltos'' que se le permiten al paquete para llegar a destino.
Cuando un router recibe un paquete para redireccionarlo, previo a esto decrementa en 1 su TTL, si el mismo resulta igual a 0 entonces devuelve el mensaje \textbf{ICMP Time Exceeded} a la dirección de origen.

La herramienta se basa en este comportamiento para obtener la información de la ruta. La misma envía varios paquetes incrementando el TTL progresivamente desde 1 hasta que el paquete efectivamente llega a destino, y va registrando los paquetes de respuesta \textbf{ICMP Time Exceeded} que recibe de cada router.

Esta herramienta se puede implementar con paquetes ICMP, TCP o UDP, según necesidad. La herramienta default que se ofrece en cualquier distribución de Linux permite realizar el procedimiento con los tres tipos de paquetes.
