Para poder entender el trabajo que hicimos es preciso en primer lugar comprender
el funcionamiento del protocolo ARP. El protocolo se utiliza para averiguar a
qué dirección MAC corresponde una dirección IP determinada dentro de una red. El
nodo que quiere averiguar esta información hace un broadcast Ethernet de un
paquete ARP ``who-has'' con la dirección IP a consultar, y alguno de los nodos
de la red que tenga la respuesta en su tabla ARP le envía un paquete ARP
``is-at'' con la dirección MAC correspondiente.

La herramienta que desarrollamos captura los paquetes Ethernet que circulen en la red mediante el comando \textit{sniff()} del paquete Scapy.
Por cada paquete guardamos su campo ``type'' y su tamaño, para luego poder calcular la fuente $S$ y el overhead impuesto por el protocolo ARP.
Además, para los paquetes ARP en particular guardamos los siguientes datos: la IP destino (de qué dispositivo se quiere averiguar la MAC), la IP origen (qué dispositivo es el que quiere averiguarla), y el tipo de paquete (who-has o is-at).

El primer desafío fue elegir el campo de los paquetes ARP a usar como fuente $S_1$. Las posibilidades eran:
\begin{itemize}
	\item Quién emite un paquete who-has (\textit{who-has-src})
    \item Por quién pregunta un paquete who-has (\textit{who-has-dst})
    \item Quién responde un paquete is-at (\textit{is-at-src})
    \item A quién le responde un paquete is-at (\textit{is-at-dst})
\end{itemize}

Para elegirla, realizamos las capturas y analizamos la información obtenida en cada caso. Eso nos llevó a elegir \textit{who-has-dst}. Descartamos ambos \textit{is-at} porque en las redes cableadas no veíamos practicamente tráfico de estos paquetes (solo los emitidos y los respondidos por la propia computadora que realizaba las capturas) dado que por defecto no se \textit{floodean}. Dentro de los (\textit{who-has}), preferimos ver por quién se pregunta, por ser en general más consistente con nuestro conocimiento sobre las redes analizadas (al parecer, algunos nodos preguntan bastante comparativamente, por más que no sean particularmente relevantes en la red).

Como requería el trabajo, implementamos una herramienta para capturar paquetes y realizamos capturas en cuatro puntos distintos: los laboratorios del Departamento de Computación, la biblioteca del Pabellón 2, un Starbucks ubicado en Av. Cabildo y Juana Azurduy y la casa de uno de los integrantes.


%ME FALTA EL CÁLCULO DE ENTROPÍA DE LAS FUENTES.
